\begin{resumo}
Com o objetivo de mensurar a eficiência através da Análise Envoltória de Dado (DEA) este estudo tem o objetivo de avaliar o quantitativos de formandos por departamento da Universidade de Brasília em um dado período. A fim de representar e calcular a eficiência por departamentos na formação de graduados e pós graduados na Universidade. Tal trabalho permitirá demonstrar a eficiência e buscará representar a capacidade dos departamentos na formação acadêmica de seus discentes. Os dados serão extraídos de sistemas institucionais da Universidade de Brasília, apropriando de 15 (quinze) departamentos que serão definidos como unidades tomadoras de decisão (DMU), viabilizando a comparação de eficiência e demostrando informações para análise.
Contudo, a modelagem irá ser compostas de entrada e saídas que irão servir de insumos para o modelo de análise apresentada, que por sua vez, trará uma representação do modelo de eficiência para observação.

 \vspace{\onelineskip}
    
 \noindent
 \textbf{Palavras-chaves}: Análise Envoltória de Dado. Universidade. Eficiência. Discentes Formandos.
\end{resumo}
