\chapter*[Introdução]{Introdução}
\addcontentsline{toc}{chapter}{Introdução}

É possível observar que nos útimos anos existe um crescente interesse por DEA (Data Envelopment Analysis) e os problemas de programação linear envolvidos. Devido à flexibilidade e à aplicabilidade na análise da eficiência em organizações de mercado e instituições públicas. Observa-se aplicação dos modelos clássicos existentes, o modelo CCR \citeonline{charnes1978measuring} e o modelo BCC \citeonline{banker1984some}, com a finalidade de ter conhecimento dos índices de eficiência, benchmarks e alvos. 

A Análise Envoltória de Dados (\textit{Data Envelopment Analysis} – DEA) é um modelo de calculo matemático com o objetivo de mensurar a eficiência que utiliza problemas de programação linear (PPLs) avaliando assim o desempenho das unidades tomadoras de decisão (\textit{Decision Making Units} – DMUs).


Em aspectos gerais o objetivo desse trabalho é abordar a programação linear e analisar o desempenho de unidades departamentais da Universidade de Brasília, definidas na literatura de Unidades de Tomadas de Decisão, considerando informações de insumos e produtos.

A abordagem dessa programação linear tem a intenção de conceber dois objetivos fundamentais: construir
fronteiras de produção a partir de dados e computar uma medida de produtividade relacionando se dados de observação com as fronteiras de produção. Verificando os pontos observados, que são combinações de insumos e produtos de um conjunto de unidades, estabelecendo assim, como medida de eficiência. 

Tendo essa referência será possível comparar a eficiência ou a ineficiência, das várias Unidades de Tomadas de Decisão em comparação com a fronteira de produção.








